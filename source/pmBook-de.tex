% This file is part of proMusicologica.ltx
% (c) 2022 Karsten Reincke (https://github.com/kreincke/proMusicologica.ltx)
% It is distributed under the terms of the creative commons license
% CC-BY-4.0 (= https://creativecommons.org/licenses/by/4.0/)

\documentclass[
  DIV=calc,
  BCOR=5mm,
  11pt,
  headings=small,
  oneside,
  abstract=true,
  toc=bib,
  english,ngerman]{scrbook}

%%% (1) general configurations %%%
\usepackage[utf8]{inputenc}
\usepackage{a4}
\usepackage[english,ngerman]{babel}

\usepackage[
  backend=biber,
  style=authortitle-dw,
  sortlocale=auto,
]{biblatex}
\ProvidesFile{biblatex.cfg}
% This file is part of proMusicologica.ltx
% (c) 2022 Karsten Reincke (https://github.com/kreincke/proMusicologica.ltx)
% It is distributed under the terms of the creative commons license
% CC-BY-4.0 (= https://creativecommons.org/licenses/by/4.0/)


\ExecuteBibliographyOptions{
  %%%%%%%%%%%%%%%%%%%%%%%%%%%%%%%%%%%%%%%%%%%%
  % configurations offered by biblatex itself
  % ------------------------------------------
  maxnames=5,  % Truncate author list after 5 authors ...
  minnames=3,  % ... But display at least 3 authors
  autocite=inline,
  hyperref=true,  % Use hyperref package (should be automatically detected, though)
  backref=true,  % Back references from bibliography page to each encounter
  backrefstyle=two,  % Combine back refs if on two consecutive pages
  isbn=true,  % (Dont) print ISBN, ISSN numbers
  autolang=hyphen,
  % track 'reused' csquotes
  citetracker=constrict, %
  loccittracker=constrict, % discriminate different pages (a.a.O) versues same page (ebda)
  opcittracker=constrict,
  idemtracker=constrict,
  ibidtracker=constrict,
  pagetracker=true,
  %%%%%%%%%%%%%%%%%%%%%%%%%%%%%%%%%%%%%%%%%%%%
  % configurations offered by authortitle-dw
  % ------------------------------------------
  annotation =true,
  namefont=normal,
  firstnamefont=normal,
  idemfont=italic,
  ibidemfont=italic,
  idembib=true, % cluster the books of the same author in bib
  idembibformat=idem, % indicate the same author by ders/dies.
  editorstring=brackets, % parens=(Hrsg.) | brackets=[Hrsg.] | normal = , Hrsg.
  nopublisher=false, % insert publisher into bib data
  editionstring=true, % allow strings in the edition field
  % ZNAME VOL (YEAR) Nr. YOURNALNUMBER
  journalnumber=afteryear, %
  %journumstring=h.
  series=afteryear,
  seriesformat=parens,
  addyear=true, %insert year after titel in 'shorttitle' hints => no year in shorttitle
  firstfull=true, % frist quote complete
  edstringincitations=false, %editor and translator only in the first
  citepages=separate, % vollzitat erst mit seiten, dann 'hier: S. 12'
}
% Put your definitions here.

% refine seriesformat by adding a prefix inside of the parens
\renewcommand*{\seriespunct}{=\addspace}

% by default Biblatex-dw uses Autor1/Autor2 in cites
% these redefinitions overwrite that behaviour by
% duplicating the style used for the bibliography
% (S. p. 35 in the German biblatex-dw handbook )
\renewcommand*{\citemultinamedelim}{\addcomma\space}
\renewcommand*{\citefinalnamedelim}{%
\ifnum\value{liststop}>2 \finalandcomma\fi
\addspace\bibstring{and}\space}%
\renewcommand*{\citerevsdnamedelim}{\addspace}

% biblatex-dw does not print 'ders.' / 'dies.' in a row of cites quoting
% the same book. Additionally, it does not know the diffrence between the
% same page of of the same work and a different page of the same work
% quoted before.
%
% As soon as biblatex 3.15 is offered by UBUNTU
% use \bibncpstring[\mkibid]{ibidem} instaed of inserting the
% German string literally as this hack does:
\renewbibmacro*{cite:ibid}{%
{\ifthenelse{\ifloccit}
  {\printtext[bibhyperref]{\usebibmacro{cite:idem} \mkibid{ebda}}%
   \global\booltrue{cbx:loccit}}
  {\printtext[bibhyperref]{\usebibmacro{cite:idem} \mkibid{a.a.O, }}}
}
}%
\DefineBibliographyStrings{english}{%
  urlseen = {RDL},
}
\DefineBibliographyStrings{german}{%
  urlseen = {RDL},
}
\endinput


\addbibresource{bib/lit.verify.bib}
\addbibresource{bib/lit.main.bib}

% package for improving the grey value and the line feed handling
\usepackage{microtype}

%language specific quoting signs
\usepackage[
  style=german,
  autostyle=true,
]{csquotes}

% language specific hyphenation
% This file is part of proMusicologica.ltx
% (c) 2022 Karsten Reincke (https://github.com/kreincke/proMusicologica.ltx)
% It is distributed under the terms of the creative commons license
% CC-BY-4.0 (= https://creativecommons.org/licenses/by/4.0/)

\hyphenation{ pro-scien-tia there-fo-re}

\babelhyphenation{ pro-scien-tia wor-tmi-tfalsch-entr-ennu-ngen}


%%% (3) layout page configuration %%%

% select the visible parts of a page
% S.31: { plain|empty|headings|myheadings }
\pagestyle{headings}

% select the wished style of page-numbering
% S.32: { arabic,roman,Roman,alph,Alph }
\pagenumbering{arabic}
\setcounter{page}{1}

% select the wished distances using the general setlength order:
% S.34 { baselineskip| parskip | parindent }
% - general no indent for paragraphs
\setlength{\parindent}{0pt}
\setlength{\parskip}{1.2ex plus 0.2ex minus 0.2ex}


%- start(footnote-configuration)

\deffootnote[1.5em]{1.5em}{1.5em}{\textsuperscript{\thefootnotemark)\ }}

% if document class = book: count footnotes from start to end
%\counterwithout{footnote}{chapter}
%- end(footnote-configuration)



%for using label as nameref
\usepackage{nameref}

%integrate nomenclature
% This file is part of proMusicologica.ltx
% (c) 2022 Karsten Reincke (https://github.com/kreincke/proMusicologica.ltx)
% It is distributed under the terms of the creative commons license
% CC-BY-4.0 (= https://creativecommons.org/licenses/by/4.0/)

\usepackage[intoc]{nomencl}
\let\abbr\nomenclature
% Deutsche Überschrift
%\renewcommand{\nomname}{Abbreviations}
\renewcommand{\nomname}{Abkürzungen}

\setlength{\nomlabelwidth}{.20\hsize}
\renewcommand{\nomlabel}[1]{#1 \dotfill}
% reduce the line distance
\setlength{\nomitemsep}{-\parsep}
\makenomenclature


% Hyperlinks
\usepackage{hyperref}
\hypersetup{bookmarks=true,breaklinks=true,colorlinks=true,citecolor=blue,draft=false}
\usepackage{longtable}

% This file is part of proMusicologica.ltx
% (c) 2022 Karsten Reincke (https://github.com/kreincke/proMusicologica.ltx)
% It is distributed under the terms of the creative commons license
% CC-BY-4.0 (= https://creativecommons.org/licenses/by/4.0/)

\usepackage{musicography}
\usepackage{harmony}

\newcommand{\tbsl}[1]{\textbackslash{#1}}
% Unfortunately not only musixtex, but also musicography and/or harmony
% still use outdated commands. So, we must 'redefine' these outdated commands:
\makeatletter
\DeclareOldFontCommand{\rm}{\normalfont\rmfamily}{\mathrm}
\DeclareOldFontCommand{\sf}{\normalfont\sffamily}{\mathsf}
\DeclareOldFontCommand{\tt}{\normalfont\ttfamily}{\mathtt}
\DeclareOldFontCommand{\bf}{\normalfont\bfseries}{\mathbf}
\DeclareOldFontCommand{\it}{\normalfont\itshape}{\mathit}
\DeclareOldFontCommand{\sl}{\normalfont\slshape}{\@nomath\sl}
\DeclareOldFontCommand{\sc}{\normalfont\scshape}{\@nomath\sc}
\makeatother



\begin{document}

%% use all entries of the bliography
\nocite{*}

\titlehead{Klassifikation}
\subject{Release \input{cfg/inc.rel.tex}}
\title{Titel}
\subtitle{Untertitel}
\author{Autor% This file is part of proMusicologica.ltx
% (c) 2022 Karsten Reincke (https://github.com/kreincke/proMusicologica.ltx)
% It is distributed under the terms of the creative commons license
% CC-BY-4.0 (= https://creativecommons.org/licenses/by/4.0/)

\footnote{\textbf{Dieser Text wird unter der XYZ Lizenz veröffentlicht.}
Hier können Ihre Bedingungen stehen, unter denen Sie Ihren Text weitergeben.
Gute Kandidaten wären z.B. die Creative Commons Lizenzen
\href{https://creativecommons.org/}{https://creativecommons.org/}. Traditionell ist auch die Formel: \emph{Alle Rechte vorbehalten. Die Verwendung von Text und Bildern, auch auszugsweise, bedarf der schriftlichen Zustimmung}. \newline
Da Ihre Arbeit auf dem Templatesystem \textit{proMusicologica.ltx} aufbaut und da Sie dieses unter den Bedinungen der \texttt{CC BY 4.0} Lizenz erhalten haben, müssen Sie auf dessen Verwendung hinweisen. Eine lizenzerfüllende Notiz könnte sein:
\newline
\textit{Erstellt auf der Basis des CC-BY-4.0 lizenzierten Tools \texttt{proMusicologica} von K. Reincke \copyright{} 2022 [
Repository \href{https://github.com/kreincke/proMusicologica.ltx}{https://github.com/kreincke/proMusicologica.ltx} ,
Lizenztext \href{https://creativecommons.org/licenses/by/4.0/}{https://creativecommons.org/licenses/by/4.0/} ]
}}
}

\maketitle
\footnotesize
\tableofcontents
\normalsize

\section{Über den Zweck dieses Textes}
\begin{quote}
\itshape
Dies ist ein \emph{\LaTeX}, \emph{Bib\LaTeX}, \emph{proScientia.ltx} und \emph{Lilypond} basiertes Template aus \emph{proMusicologica.ltx} für leicht zu scheibende musikwissenschaftliche Bücher. Es demonstriert seinen Zweck mit einer Zitatverifikation und integrierten Inline- und autonomen Notenbeispielen -- nur, damit Sie das durch Ihren eigenen Text ersetzen.
\end{quote}

\chapter{Zur Integration von Zitaten und Belegen}

\section{Verifikation: Zitatmarkierung}

\begin{itemize}

  \item \enquote{Zitat mit \enquote{eingebettetem Zitat}}

  \item Deutscher Satz \foreignquote{german}{with embedded foreign phrase}
  als eingebettetes Zitat. Erwartetes Resultat: Deutsche Anführungszeichen,
  weil Teil in einem deutsche Satz.

  \item Autonomes englischsprachiges Zitat:
  \begin{quote}
    \foreignquote{english}{This shall be an English written paragraph containing
    a set of sentences which together build the quote.}
  \end{quote}
  Erwartetes Ergebnis: Englische Anführungszeichen, weil autonomer Satz.
\end{itemize}

\section{Verifikation: Literaturnachweis}
\begin{itemize}
  \item Buch1 (erstverwendungen)\footcite[vgl.][15]{Grabner1974a}
  \item ders. ebda.\footcite[vgl.][15]{Grabner1974a}
  \item ders. a.a.O.\footcite[vgl.][23]{Grabner1974a}
  \item Buch2 (erstverwendung)\footcite[vgl.][15]{Delamotte2011a}
  \item Buch1 (wiederverwendung)\footcite[vgl.][15]{Grabner1974a}
\end{itemize}

\chapter{Integration von Snippets}
\begin{quote}\itshape
In diesem Kapitel erläutern wir \ldots
\end{quote}

\section{Snippetdemo}
% This file is part of proMusicologica.ltx
% (c) 2022 Karsten Reincke (https://github.com/kreincke/proMusicologica.ltx)
% It is distributed under the terms of the creative commons license
% CC-BY-4.0 (= https://creativecommons.org/licenses/by/4.0/)

\subsection{snippet section 1}
\subsection{snippet section 2}
\subsection{snippet section 3}


\chapter{Musik-Integration}

\section{Im Fließtext verwendbare Notensymbole}

\begin{scriptsize}
\begin{longtable}{|c|l|l|}
\hline
 & Command & From\\
\hline
\hline
\Takt{2}{2} & \texttt{\tbsl{Takt}\{2\}\{2\}} & \tbsl{usepackage\{harmony\}} \\
\hline
\Takt{3}{4} & \texttt{\tbsl{Takt}\{3\}\{4\}} & \tbsl{usepackage\{harmony\}} \\
\hline
\Takt{4}{4} & \texttt{\tbsl{Takt}\{4\}\{4\}} & \tbsl{usepackage\{harmony\}} \\
\hline
 & \ldots & \tbsl{usepackage\{harmony\}} \\
\hline
\Takt{c}{0} & \texttt{\tbsl{Takt}\{c\}\{0\}} & \tbsl{usepackage\{harmony\}} \\
\hline
\Takt{c}{1} & \texttt{\tbsl{Takt}\{c\}\{1\}} & \tbsl{usepackage\{harmony\}} \\
\hline
\meterCutC & \texttt{\tbsl{meterCutC}}& \tbsl{usepackage\{musicography\}} \\
\hline
$\sharp$ & \texttt{\$\tbsl{sharp}\$} & Mathematikumgebung \\
\hline
\musSharp & \texttt{\tbsl{musSharp}} & \tbsl{usepackage\{musicography\}} \\
\hline
\musDoubleSharp & \texttt{\tbsl{musDoubleSharp}} & \tbsl{usepackage\{musicography\}} \\
\hline
$\flat$ & \texttt{\$\tbsl{flat}\$} & Mathematikumgebung \\
\hline
\musFlat & \texttt{\tbsl{musFlat}} & \tbsl{usepackage\{musicography\}} \\
\hline
\musDoubleFlat & \texttt{\tbsl{musDoubleFlat}} & \tbsl{usepackage\{musicography\}} \\
\hline
$\natural$ & \texttt{\$\tbsl{natural}\$} & Mathematikumgebung \\
\hline
\musNatural & \texttt{\tbsl{musNatural}} & \tbsl{usepackage\{musicography\}} \\
\hline
\musWhole & \texttt{\tbsl{musWhole}} & \tbsl{usepackage\{musicography\}} \\
\hline
\Ganz & \texttt{\tbsl{Ganz}} & \tbsl{usepackage\{harmony\}} \\
\hline
\musHalfDotted & \texttt{\tbsl{musHalfDotted}} & \tbsl{usepackage\{musicography\}} \\
\hline
\Halb\Pu & \texttt{\tbsl{Halb}\tbsl{Pu}} & \tbsl{usepackage\{harmony\}} \\
\hline
\musHalf & \texttt{\tbsl{musHalf}} & \tbsl{usepackage\{musicography\}} \\
\hline
\Halb & \texttt{\tbsl{Halb}} & \tbsl{usepackage\{harmony\}} \\
\hline
\musQuarterDotted & \texttt{\tbsl{musQuarterDotted}}& \tbsl{usepackage\{musicography\}} \\
\hline
\Vier\Pu & \texttt{\tbsl{Vier}\tbsl{Pu}} & \tbsl{usepackage\{harmony\}} \\
\hline
\musQuarter & \texttt{\tbsl{musQuarter}} & \tbsl{usepackage\{musicography\}} \\
\hline
\Vier & \texttt{\tbsl{Vier}} & \tbsl{usepackage\{harmony\}} \\
\hline
\Acht\Pu & \texttt{\tbsl{Acht}\tbsl{Pu}} & \tbsl{usepackage\{harmony\}} \\
\hline
\musEighth & \texttt{\tbsl{musEighth}} & \tbsl{usepackage\{musicography\}} \\
\hline
\Acht & \texttt{\tbsl{Acht}} & \tbsl{usepackage\{harmony\}} \\
\hline
\AchtBL & \texttt{\tbsl{AchtBL}}\footnote{Werden die Elemente \emph{AchtBL} (= \AchtBL), \emph{AchtBR} (= \AchtBR\ ), \emph{SechBL} (= \SechBL) und \emph{SechBR} (= \SechBR\ ) mit den anderen Elementen geeignet verbunden, entstehen sogar rhythmische Ketten:
\Takt{c}{0} \Vier\ \Vier\AchtBL\ \Vier\Pu\ \Acht\ $|$
\AchtBR\Pu \SechBl \AchtBR\kern-0.15em\SechBR\Vier\ \SechBr\Vier\SechBl\ $|$ \
=
\tbsl{Vier}\tbsl{ }\tbsl{Vier}\tbsl{AchtBL}\tbsl{ }\tbsl{Vier}\tbsl{Pu}
\tbsl{ }\tbsl{Acht}\tbsl{ }\$\textbar\$
\tbsl{AchtBR}\tbsl{Pu} \tbsl{SechBl}
\tbsl{AchtBR}\tbsl{kern-0.15em}\tbsl{SechBR}\tbsl{Vier}\tbsl{ }
\tbsl{SechBr}\tbsl{Vier}\tbsl{SechBl}\tbsl{ }\$\textbar\$
 } & \tbsl{usepackage\{harmony\}} \\
\hline
\AchtBR & \texttt{\tbsl{AchtBR}}  & \tbsl{usepackage\{harmony\}} \\
\hline
\Vier\AchtBL & \texttt{\tbsl{Vier} \tbsl{AchtBL}} & \tbsl{usepackage\{harmony\}} \\
\hline
\Sech\Pu & \texttt{\tbsl{Sech}\tbsl{Pu}} & \tbsl{usepackage\{harmony\}} \\
\hline
\musSixteenth & \texttt{\tbsl{musSixteenth}} & \tbsl{usepackage\{musicography\}} \\
\hline
\Sech & \texttt{\tbsl{Sech}} & \tbsl{usepackage\{harmony\}} \\
\hline
\SechBL & \texttt{\tbsl{SechBL}} & \tbsl{usepackage\{harmony\}} \\
\hline
\SechBR & \texttt{\tbsl{SechBR}} & \tbsl{usepackage\{harmony\}} \\
\hline
\Vier\SechBL & \texttt{\tbsl{Vier} \tbsl{SechBL} } & \tbsl{usepackage\{harmony\}} \\
\hline
\Zwdr & \texttt{\tbsl{Zwdr}} & \tbsl{usepackage\{harmony\}} \\
\hline
\GaPa & \texttt{\tbsl{GaPa}} & \tbsl{usepackage\{harmony\}} \\
\hline
\HaPa & \texttt{\tbsl{HaPa}} & \tbsl{usepackage\{harmony\}} \\
\hline
\ViPa & \texttt{\tbsl{ViPa}} & \tbsl{usepackage\{harmony\}} \\
\hline
\AcPa & \texttt{\tbsl{AcPa}} & \tbsl{usepackage\{harmony\}} \\
\hline
\SePa & \texttt{\tbsl{SePa}} & \tbsl{usepackage\{harmony\}} \\
\hline
\ZwPa & \texttt{\tbsl{ZwPa}} & \tbsl{usepackage\{harmony\}} \\
\hline
\hline
\end{longtable}
\end{scriptsize}

\section{Im Fließtext verwendbare Harmonieanalysesymbole}

\begin{scriptsize}
\begin{longtable}{|c|l|l|}
\hline
 & Command & From \\
\hline
\hline
\HH.T..... & \texttt{\tbsl{HH.T.....}} & \tbsl{usepackage\{harmony\}} \\
\hline
\HH.Tp..... & \texttt{ \tbsl{HH.Tp.....}} & \tbsl{usepackage\{harmony\}} \\
\hline
\HH.S..... & \texttt{\tbsl{HH.S.....}} & \tbsl{usepackage\{harmony\}} \\
\hline
\HH.D..... & \texttt{\tbsl{HH.D.....}} & \tbsl{usepackage\{harmony\}} \\
\hline
 & \ldots & \tbsl{usepackage\{harmony\}} \\
\hline
\HH.D.3.9.7..  & \texttt{\tbsl{HH.D.3.9.7..}} & \tbsl{usepackage\{harmony\}} \\
\hline
\HH.T..9$\flat\rightarrow$8.7..  & \verb|\HH.T..9$\flat\rightarrow$8.7..| & \tbsl{usepackage\{harmony\}} \\
\hline
\Dohne & \texttt{\tbsl{Dohne}} & \tbsl{usepackage\{harmony\}} \\
\hline
\DD & \texttt{\tbsl{DD}} & \tbsl{usepackage\{harmony\}} \\
\hline
\DS & \texttt{\tbsl{DS}} & \tbsl{usepackage\{harmony\}} \\
\hline
\HH.\DD.5\VM.7... & \texttt{\tbsl{HH}.\tbsl{DD}.5\tbsl{VM}.7...} & \tbsl{usepackage\{harmony\}} \\
\hline
\HH.I..\texttt{(5)}.\texttt{(3)}.. & \verb|\HH.I..\texttt{(5)}.\texttt{(3)}..| & \tbsl{usepackage\{harmony\}} \\
\hline
\HH.III..\texttt{ 6 }.\texttt{(3)}.. & \verb|\HH.III..\texttt{ 6 }.\texttt{(3)}..| & \tbsl{usepackage\{harmony\}} \\
\hline
\HH.V..\texttt{ 6}.\texttt{ 4}.. & \verb|\HH.V..\texttt{ 6}.\texttt{ 4}..| & \tbsl{usepackage\{harmony\}} \\
\hline
\HH.IV..\texttt{(5)}.\texttt{ 3$\flat$ }.. & \verb|\HH.IV..\texttt{(5)}.\texttt{ 3$\flat$ }.. | & \tbsl{usepackage\{harmony\}} \\
\hline
\HH.V..\texttt{ 6 }.\texttt{ 3$\flat$}.. & \verb|\HH.V..\texttt{ 6 }.\texttt{ 3$\flat$}..| & \tbsl{usepackage\{harmony\}} \\
\hline
\HH.V..\texttt{ 6$\rightarrow$7}.\texttt{ 5 }.\texttt{(3)}. & \verb|\HH.V..\texttt{6$\rightarrow$7}.\texttt{ 5 }.\texttt{(3)}.| & \tbsl{usepackage\{harmony\}} \\
\hline
\hline
\end{longtable}
\end{scriptsize}

\section{Autonomes Notenbeispiel}

\begin{center}
\begin{lilypond}
\version "2.18.2"

\header { tagline = "" }
\include "lilypond/harmonyli.ly"

\score {
  \new StaffGroup {
    \time 4/4
    <<
      \new Staff {
        \relative d' {
       	  \clef "treble" \key d \major  \stemUp
          < fis d'>2 < fis dis'>2 < b e>2 < b eis>2 |
          < b fis'>2 < e gis >2 < e a >4 < e a >4 < a fis>2  \bar "||"
        }
        \addlyrics {
          \markup \setHas "T" #'(("C"."D")("fr" . " "))
          \markup \setImHas "D" #'(("B"."1")("a" . "7")("fr" . " "))
          \markup \setHas "Sp" #'(("B"."7")("a" . "7")("fl" . " ")("fr" . " "))
          \markup \setHas "D" #'(("T"."x")("B"."3")("a" . "5")("b" . "7")("c" . "♭9>♯8")("fr" . " "))
          \markup \setHas "Tp" #'(("B"."3")("fl" . " ")("fr" . " "))
          \markup \setHas "D" #'(("T"."d")("B"."5")("a" . "7")("b" . "8")("fr" . " "))
          \initTextSpan "   "
          \markup \openZoomRow "D" #'(("a"."4")("fl" . " "))
          \startTextSpan
          \markup \expZoomRow #'(("a"."3")("fr" . " "))
          \stopTextSpan
          \markup \setHas "T" #'(("fr" . " "))
        }
      }
      \new Staff {
        \relative d {
          \clef "bass" \key d \major \stemDown
          < d a'>2 < b a'>2 < d g>2 < cis g'>2 |
          < d fis>2 < b d>2 << { a4( a4) } { d4( cis4) } >> < d, d'>2 \bar "||"
        }
      }
    >>
  }
  \layout {
    \context { \Lyrics \consists "Text_spanner_engraver" }
  }
  \midi {}
}

\end{lilypond}
\end{center}


% This file is part of proMusicologica.ltx
% (c) 2022 Karsten Reincke (https://github.com/kreincke/proMusicologica.ltx)
% It is distributed under the terms of the creative commons license
% CC-BY-4.0 (= https://creativecommons.org/licenses/by/4.0/)


% specific abbreviations
\abbr[utb]{UTB}{Uni-Taschenbuch}
\abbr[stw]{stw}{suhrkamp taschenbuch wissenschaft}

% general abbreviations
\abbr[vgl]{vgl.}{vergleiche}
\abbr[aaO]{a.a.O.}{am angegebenen Ort}
\abbr[ebda]{ebda.}{ebenda}
\abbr[ifross]{ifross}{Institut für Rechtsfragen der Freien und Open Source Software}
\abbr[wp]{wp.}{webpage = Webdokument ohne innere Seitennummerierung}
\abbr[RDL]{RDL}{Referenzdownload am}

% This file is part of proMusicologica.ltx
% (c) 2022 Karsten Reincke (https://github.com/kreincke/proMusicologica.ltx)
% It is distributed under the terms of the creative commons license
% CC-BY-4.0 (= https://creativecommons.org/licenses/by/4.0/)

\abbr[fzmw]{FZMw}{Frankfurter Zeitung für Musikwissenschaft}

\abbr[afmw]{AfMW}{Archiv für Musikwissenschaft [ISSN: 0003-9292]}
\abbr[dtk]{}{Die Tonkunst [ISSN: 1863-3536]}

\printnomenclature
\printbibliography

\end{document}
