% This file is part of proMusicologica.ltx
% (c) 2022 Karsten Reincke (https://github.com/kreincke/proMusicologica.ltx)
% It is distributed under the terms of the creative commons license
% CC-BY-4.0 (= https://creativecommons.org/licenses/by/4.0/)

\documentclass[
  DIV=calc,
  BCOR=5mm,
  11pt,
  headings=small,
  oneside,
  abstract=true,
  toc=bib,
  ngerman,english]{scrartcl}

%%% (1) general configurations %%%
\usepackage[utf8]{inputenc}
\usepackage{a4}
\usepackage[ngerman,english]{babel}

\usepackage[
  backend=biber,
  style=authortitle-dw,
  sortlocale=auto,
]{biblatex}
\ProvidesFile{biblatex.cfg}

% This file is part of proMusicologica.ltx
% (c) 2022 Karsten Reincke (https://github.com/kreincke/proMusicologica.ltx)
% It is distributed under the terms of the creative commons license
% CC-BY-4.0 (= https://creativecommons.org/licenses/by/4.0/)

\ExecuteBibliographyOptions{
  %%%%%%%%%%%%%%%%%%%%%%%%%%%%%%%%%%%%%%%%%%%%
  % configurations offered by biblatex itself
  % ------------------------------------------
  maxnames=5,  % Truncate author list after 5 authors ...
  minnames=3,  % ... But display at least 3 authors
  autocite=inline,
  hyperref=true,  % Use hyperref package (should be automatically detected, though)
  backref=true,  % Back references from bibliography page to each encounter
  backrefstyle=two,  % Combine back refs if on two consecutive pages
  isbn=true,  % (Dont) print ISBN, ISSN numbers
  autolang=hyphen,
  % track 'reused' csquotes
  citetracker=constrict, %
  loccittracker=constrict, % discriminate different pages (a.a.O) versues same page (ebda)
  opcittracker=constrict,
  idemtracker=constrict,
  ibidtracker=constrict,
  pagetracker=true,
  %%%%%%%%%%%%%%%%%%%%%%%%%%%%%%%%%%%%%%%%%%%%
  % configurations offered by authortitle-dw
  % ------------------------------------------
  annotation =true,
  namefont=normal,
  firstnamefont=normal,
  idemfont=italic,
  ibidemfont=italic,
  idembib=true, % cluster the books of the same author in bib
  idembibformat=idem, % indicate the same author by ders/dies.
  editorstring=brackets, % parens=(Hrsg.) | brackets=[Hrsg.] | normal = , Hrsg.
  nopublisher=false, % insert publisher into bib data
  editionstring=true, % allow strings in the edition field
  % ZNAME VOL (YEAR) Nr. YOURNALNUMBER
  journalnumber=afteryear, %
  %journumstring=h.
  series=afteryear,
  seriesformat=parens,
  addyear=true, %insert year after titel in 'shorttitle' hints => no year in shorttitle
  firstfull=true, % frist quote complete
  edstringincitations=false, %editor and translator only in the first
  citepages=separate, % vollzitat erst mit seiten, dann 'hier: S. 12'
}
% Put your definitions here.

% refine seriesformat by adding a prefix inside of the parens
\renewcommand*{\seriespunct}{=\addspace}

% by default Biblatex-dw uses Autor1/Autor2 in cites
% these redefinitions overwrite that behaviour by
% duplicating the style used for the bibliography
% (S. p. 35 in the German biblatex-dw handbook )
\renewcommand*{\citemultinamedelim}{\addcomma\space}
\renewcommand*{\citefinalnamedelim}{%
\ifnum\value{liststop}>2 \finalandcomma\fi
\addspace\bibstring{and}\space}%
\renewcommand*{\citerevsdnamedelim}{\addspace}

% biblatex-dw does not print 'ders.' / 'dies.' in row cites of the same book
% Additionally, it does not know the diffrence between the same page of
% of the same work and a different page of the same work quoted before.
%
% as soon as biblatex 3.15 is offered by UBUNTU use \bibncpstring[\mkibid]{ibidem}
% instaed of inserting the German string literally as this hack does:
\renewbibmacro*{cite:ibid}{%
{\ifthenelse{\ifloccit}
  {\printtext[bibhyperref]{ \mkibid{idem} \mkibid{ibid}}%
   \global\booltrue{cbx:loccit}}
  {\printtext[bibhyperref]{ \mkibid{idem} \mkibid{loc.cit., }}}
}
}%
\DefineBibliographyStrings{english}{%
  urlseen = {RDL},
}
\DefineBibliographyStrings{german}{%
  urlseen = {RDL},
}
\endinput


\addbibresource{bib/lit.verify.bib}
\addbibresource{bib/lit.main.bib}

% package for improving the grey value and the line feed handling
\usepackage{microtype}

%language specific quoting signs
\usepackage[
  style=german,
  autostyle=true,
]{csquotes}

% language specific hyphenation
% This file is part of proMusicologica.ltx
% (c) 2022 Karsten Reincke (https://github.com/kreincke/proMusicologica.ltx)
% It is distributed under the terms of the creative commons license
% CC-BY-4.0 (= https://creativecommons.org/licenses/by/4.0/)

\hyphenation{ pro-scien-tia there-fo-re}

\babelhyphenation{ pro-scien-tia wor-tmi-tfalsch-entr-ennu-ngen}


%%% (3) layout page configuration %%%

% select the visible parts of a page
% S.31: { plain|empty|headings|myheadings }
\pagestyle{headings}

% select the wished style of page-numbering
% S.32: { arabic,roman,Roman,alph,Alph }
\pagenumbering{arabic}
\setcounter{page}{1}

% select the wished distances using the general setlength order:
% S.34 { baselineskip| parskip | parindent }
% - general no indent for paragraphs
\setlength{\parindent}{0pt}
\setlength{\parskip}{1.2ex plus 0.2ex minus 0.2ex}


%- start(footnote-configuration)

\deffootnote[1.5em]{1.5em}{1.5em}{\textsuperscript{\thefootnotemark)\ }}

% if document class = book: count footnotes from start to end
%\counterwithout{footnote}{chapter}
%- end(footnote-configuration)



%for using label as nameref
\usepackage{nameref}

%integrate nomenclature
% This file is part of proMusicologica.ltx
% (c) 2022 Karsten Reincke (https://github.com/kreincke/proMusicologica.ltx)
% It is distributed under the terms of the creative commons license
% CC-BY-4.0 (= https://creativecommons.org/licenses/by/4.0/)

\usepackage[intoc]{nomencl}
\let\abbr\nomenclature
\renewcommand{\nomname}{Abbreviations}

\setlength{\nomlabelwidth}{.20\hsize}
\renewcommand{\nomlabel}[1]{#1 \dotfill}
% reduce the line distance
\setlength{\nomitemsep}{-\parsep}
\makenomenclature


% Hyperlinks
\usepackage{hyperref}
\hypersetup{bookmarks=true,breaklinks=true,colorlinks=true,citecolor=blue,draft=false}
\usepackage{longtable}

% This file is part of proMusicologica.ltx
% (c) 2022 Karsten Reincke (https://github.com/kreincke/proMusicologica.ltx)
% It is distributed under the terms of the creative commons license
% CC-BY-4.0 (= https://creativecommons.org/licenses/by/4.0/)

\usepackage{musicography}
\usepackage{harmony}

\newcommand{\tbsl}[1]{\textbackslash{#1}}
% Unfortunately not only musixtex, but also musicography and/or harmony
% still use outdated commands. So, we must 'redefine' these outdated commands:
\makeatletter
\DeclareOldFontCommand{\rm}{\normalfont\rmfamily}{\mathrm}
\DeclareOldFontCommand{\sf}{\normalfont\sffamily}{\mathsf}
\DeclareOldFontCommand{\tt}{\normalfont\ttfamily}{\mathtt}
\DeclareOldFontCommand{\bf}{\normalfont\bfseries}{\mathbf}
\DeclareOldFontCommand{\it}{\normalfont\itshape}{\mathit}
\DeclareOldFontCommand{\sl}{\normalfont\slshape}{\@nomath\sl}
\DeclareOldFontCommand{\sc}{\normalfont\scshape}{\@nomath\sc}
\makeatother


\begin{document}

%% use all entries of the bliography
\nocite{*}

\titlehead{Classification}
\subject{Release 0.1}
\title{Title}
\subtitle{Subtitle}
\author{Autor% This file is part of proMusicologica.ltx
% (c) 2022 Karsten Reincke (https://github.com/kreincke/proMusicologica.ltx)
% It is distributed under the terms of the creative commons license
% CC-BY-4.0 (= https://creativecommons.org/licenses/by/4.0/)


\footnote{\textbf{This text is distributed under the terms of the XYZ license.} Please insert your conditions here - for example by choosing a Creative Commons License \href{https://creativecommons.org/}{https://creativecommons.org/}. Alternatively, one often sees \emph{All rights reserved}. \newline
Since your work is based on the template system \textit{proMusicologica.ltx} and since you use the system under the terms of the \texttt{CC-BY-4.0} license, you must acknowledge that you have used it. A corresponding license-fulfilling note could be:\newline
Developed on the base of the \texttt{CC-BY-4.0} licensed tool \textit{proMusicologica.ltx} by K. Reincke \copyright{} 2022 [
repository = \href{https://github.com/kreincke/proMusicologica.ltx}{https://github.com/kreincke/proMusicologica.ltx},
license text = \href{https://creativecommons.org/licenses/by/4.0/}{https://creativecommons.org/licenses/by/4.0/} ]
}
}

\maketitle
\begin{abstract}
\noindent \itshape
This is a template for easily writing a musicological article: Replace the given content demonstrating that your system works, by your own text.
\end{abstract}


\footnotesize
\tableofcontents
\normalsize

\section{Context-Sensitive Quotation Test}
\begin{itemize}
  \item \enquote{complete data} = First initial quotation\footcite[vgl.][123]{Grabner1974a}
  \item \enquote{idem ibid.} = same work, same page as before\footcite[cf.][123]{Grabner1974a}
  \item \enquote{idem loc. cit., p.} = same work as before, different page\footcite[cf.][125f]{Grabner1974a}
  \item \enquote{complete data} = Second initial quotation\footcite[vgl.][123]{Delamotte2011a}
  \item \enquote{Shorttitel} = first work, first page, different context\footcite[cf.][123]{Grabner1974a}
\end{itemize}

\section{Music Integration Test}

Inline usable symbols = \Takt{4}{4} \SePa \Vier\SechBL\SechBL\ \Halb\Pu\ $|$ \HH.T..... \HH.S.3.... \HH.D.3.9.7.. \HH.T....., then a Lilypond based sheet music sample:



\begin{center}
\begin{lilypond}
\version "2.18.2"

\header { tagline = "" }
\include "lilypond/harmonyli.ly"

\score {
  \new StaffGroup {
    \time 4/4
    <<
      \new Staff {
        \relative d' {
       	  \clef "treble" \key d \major  \stemUp
          r1 | < fis a d>1 \bar "||"
        }
        \addlyrics {
          \markup \setHas "T" #'(("" . " "))
        }
      }
      \new Staff {
        \relative d {
          \clef "bass" \key d \major \stemDown
          r4 d4 fis4 a | <d, d'>1 \bar "||"
        }
      }
    >>
  }
  \layout {
    \context { \Lyrics \consists "Text_spanner_engraver" }
  }
  \midi {}
}

\end{lilypond}
\end{center}

% This file is part of proMusicologica.ltx
% (c) 2022 Karsten Reincke (https://github.com/kreincke/proMusicologica.ltx)
% It is distributed under the terms of the creative commons license
% CC-BY-4.0 (= https://creativecommons.org/licenses/by/4.0/)


\abbr[cf]{cf.}{confer}
\abbr[id]{id.}{idem = latin for 'the same', be it a man, woman or a group\ldots}
\abbr[ibid]{ibid.}{ibidem = latin for 'at the same place'}
\abbr[lc]{l.c.}{loco citato = latin for 'in the place cited'}
\abbr[wpe]{wp.}{webpage = Web document without a finer page division}
\abbr[RDL]{RDL}{Reference download on}

% This file is part of proMusicologica.ltx
% (c) 2022 Karsten Reincke (https://github.com/kreincke/proMusicologica.ltx)
% It is distributed under the terms of the creative commons license
% CC-BY-4.0 (= https://creativecommons.org/licenses/by/4.0/)

\abbr[fzmw]{FZMw}{Frankfurter Zeitung für Musikwissenschaft}

\abbr[afmw]{AfMW}{Archiv für Musikwissenschaft [ISSN: 0003-9292]}
\abbr[dtk]{}{Die Tonkunst [ISSN: 1863-3536]}

\printnomenclature
\printbibliography

\end{document}
