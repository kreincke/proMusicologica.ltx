% This file is part of proMusicologica.ltx
% (c) 2022 Karsten Reincke (https://github.com/kreincke/proMusicologica.ltx)
% It is distributed under the terms of the creative commons license
% CC-BY-4.0 (= https://creativecommons.org/licenses/by/4.0/)

\documentclass[
  DIV=calc,
  BCOR=5mm,
  11pt,
  headings=small,
  oneside,
  abstract=true,
  toc=bib,
  english,ngerman]{scrartcl}

%%% (1) general configurations %%%
\usepackage[utf8]{inputenc}
\usepackage{a4}
\usepackage[english,ngerman]{babel}

\usepackage[
  backend=biber,
  style=authortitle-dw,
  sortlocale=auto,
]{biblatex}
\ProvidesFile{biblatex.cfg}
% This file is part of proMusicologica.ltx
% (c) 2022 Karsten Reincke (https://github.com/kreincke/proMusicologica.ltx)
% It is distributed under the terms of the creative commons license
% CC-BY-4.0 (= https://creativecommons.org/licenses/by/4.0/)


\ExecuteBibliographyOptions{
  %%%%%%%%%%%%%%%%%%%%%%%%%%%%%%%%%%%%%%%%%%%%
  % configurations offered by biblatex itself
  % ------------------------------------------
  maxnames=5,  % Truncate author list after 5 authors ...
  minnames=3,  % ... But display at least 3 authors
  autocite=inline,
  hyperref=true,  % Use hyperref package (should be automatically detected, though)
  backref=true,  % Back references from bibliography page to each encounter
  backrefstyle=two,  % Combine back refs if on two consecutive pages
  isbn=true,  % (Dont) print ISBN, ISSN numbers
  autolang=hyphen,
  % track 'reused' csquotes
  citetracker=constrict, %
  loccittracker=constrict, % discriminate different pages (a.a.O) versues same page (ebda)
  opcittracker=constrict,
  idemtracker=constrict,
  ibidtracker=constrict,
  pagetracker=true,
  %%%%%%%%%%%%%%%%%%%%%%%%%%%%%%%%%%%%%%%%%%%%
  % configurations offered by authortitle-dw
  % ------------------------------------------
  annotation =true,
  namefont=normal,
  firstnamefont=normal,
  idemfont=italic,
  ibidemfont=italic,
  idembib=true, % cluster the books of the same author in bib
  idembibformat=idem, % indicate the same author by ders/dies.
  editorstring=brackets, % parens=(Hrsg.) | brackets=[Hrsg.] | normal = , Hrsg.
  nopublisher=false, % insert publisher into bib data
  editionstring=true, % allow strings in the edition field
  % ZNAME VOL (YEAR) Nr. YOURNALNUMBER
  journalnumber=afteryear, %
  %journumstring=h.
  series=afteryear,
  seriesformat=parens,
  addyear=true, %insert year after titel in 'shorttitle' hints => no year in shorttitle
  firstfull=true, % frist quote complete
  edstringincitations=false, %editor and translator only in the first
  citepages=separate, % vollzitat erst mit seiten, dann 'hier: S. 12'
}
% Put your definitions here.

% refine seriesformat by adding a prefix inside of the parens
\renewcommand*{\seriespunct}{=\addspace}

% by default Biblatex-dw uses Autor1/Autor2 in cites
% these redefinitions overwrite that behaviour by
% duplicating the style used for the bibliography
% (S. p. 35 in the German biblatex-dw handbook )
\renewcommand*{\citemultinamedelim}{\addcomma\space}
\renewcommand*{\citefinalnamedelim}{%
\ifnum\value{liststop}>2 \finalandcomma\fi
\addspace\bibstring{and}\space}%
\renewcommand*{\citerevsdnamedelim}{\addspace}

% biblatex-dw does not print 'ders.' / 'dies.' in a row of cites quoting
% the same book. Additionally, it does not know the diffrence between the
% same page of of the same work and a different page of the same work
% quoted before.
%
% As soon as biblatex 3.15 is offered by UBUNTU
% use \bibncpstring[\mkibid]{ibidem} instaed of inserting the
% German string literally as this hack does:
\renewbibmacro*{cite:ibid}{%
{\ifthenelse{\ifloccit}
  {\printtext[bibhyperref]{\usebibmacro{cite:idem} \mkibid{ebda}}%
   \global\booltrue{cbx:loccit}}
  {\printtext[bibhyperref]{\usebibmacro{cite:idem} \mkibid{a.a.O, }}}
}
}%
\DefineBibliographyStrings{english}{%
  urlseen = {RDL},
}
\DefineBibliographyStrings{german}{%
  urlseen = {RDL},
}
\endinput


\addbibresource{bib/lit.verify.bib}
\addbibresource{bib/literature.bib}

% package for improving the grey value and the line feed handling
\usepackage{microtype}

%language specific quoting signs
\usepackage[
  style=german,
  autostyle=true,
]{csquotes}

% language specific hyphenation
% This file is part of proMusicologica.ltx
% (c) 2022 Karsten Reincke (https://github.com/kreincke/proMusicologica.ltx)
% It is distributed under the terms of the creative commons license
% CC-BY-4.0 (= https://creativecommons.org/licenses/by/4.0/)

\hyphenation{ pro-scien-tia there-fo-re}

\babelhyphenation{ pro-scien-tia wor-tmi-tfalsch-entr-ennu-ngen}


%%% (3) layout page configuration %%%

% select the visible parts of a page
% S.31: { plain|empty|headings|myheadings }
\pagestyle{headings}

% select the wished style of page-numbering
% S.32: { arabic,roman,Roman,alph,Alph }
\pagenumbering{arabic}
\setcounter{page}{1}

% select the wished distances using the general setlength order:
% S.34 { baselineskip| parskip | parindent }
% - general no indent for paragraphs
\setlength{\parindent}{0pt}
\setlength{\parskip}{1.2ex plus 0.2ex minus 0.2ex}


%- start(footnote-configuration)

\deffootnote[1.5em]{1.5em}{1.5em}{\textsuperscript{\thefootnotemark)\ }}

% if document class = book: count footnotes from start to end
%\counterwithout{footnote}{chapter}
%- end(footnote-configuration)



%for using label as nameref
\usepackage{nameref}

%integrate nomenclature
% This file is part of proMusicologica.ltx
% (c) 2022 Karsten Reincke (https://github.com/kreincke/proMusicologica.ltx)
% It is distributed under the terms of the creative commons license
% CC-BY-4.0 (= https://creativecommons.org/licenses/by/4.0/)

\usepackage[intoc]{nomencl}
\let\abbr\nomenclature
% Deutsche Überschrift
%\renewcommand{\nomname}{Abbreviations}
\renewcommand{\nomname}{Abkürzungen}

\setlength{\nomlabelwidth}{.20\hsize}
\renewcommand{\nomlabel}[1]{#1 \dotfill}
% reduce the line distance
\setlength{\nomitemsep}{-\parsep}
\makenomenclature


% Hyperlinks
\usepackage{hyperref}
\hypersetup{bookmarks=true,breaklinks=true,colorlinks=true,citecolor=blue,draft=false}
\usepackage{longtable}

% This file is part of proMusicologica.ltx
% (c) 2022 Karsten Reincke (https://github.com/kreincke/proMusicologica.ltx)
% It is distributed under the terms of the creative commons license
% CC-BY-4.0 (= https://creativecommons.org/licenses/by/4.0/)

\usepackage{musicography}
\usepackage{harmony}

\newcommand{\tbsl}[1]{\textbackslash{#1}}
% Unfortunately not only musixtex, but also musicography and/or harmony
% still use outdated commands. So, we must 'redefine' these outdated commands:
\makeatletter
\DeclareOldFontCommand{\rm}{\normalfont\rmfamily}{\mathrm}
\DeclareOldFontCommand{\sf}{\normalfont\sffamily}{\mathsf}
\DeclareOldFontCommand{\tt}{\normalfont\ttfamily}{\mathtt}
\DeclareOldFontCommand{\bf}{\normalfont\bfseries}{\mathbf}
\DeclareOldFontCommand{\it}{\normalfont\itshape}{\mathit}
\DeclareOldFontCommand{\sl}{\normalfont\slshape}{\@nomath\sl}
\DeclareOldFontCommand{\sc}{\normalfont\scshape}{\@nomath\sc}
\makeatother



\begin{document}

%% use all entries of the bliography
\nocite{*}

\titlehead{Klassifikation}
\subject{Release 0.1}
\title{Musikwissenschaftliches Arbeiten}
\subtitle{Wie es leichter gehen könnte}
\author{Karsten Reincke% This file is part of proMusicologica.ltx
% (c) 2022 Karsten Reincke (https://github.com/kreincke/proMusicologica.ltx)
% It is distributed under the terms of the creative commons license
% CC-BY-4.0 (= https://creativecommons.org/licenses/by/4.0/)

\footnote{\textbf{Dieser Text wird unter der XYZ Lizenz veröffentlicht.}
Hier können Ihre Bedingungen stehen, unter denen Sie Ihren Text weitergeben.
Gute Kandidaten wären z.B. die Creative Commons Lizenzen
\href{https://creativecommons.org/}{https://creativecommons.org/}. Traditionell ist auch die Formel: \emph{Alle Rechte vorbehalten. Die Verwendung von Text und Bildern, auch auszugsweise, bedarf der schriftlichen Zustimmung}. \newline
Da Ihre Arbeit auf dem Templatesystem \textit{proMusicologica.ltx} aufbaut und da Sie dieses unter den Bedinungen der \texttt{CC BY 4.0} Lizenz erhalten haben, müssen Sie auf dessen Verwendung hinweisen. Eine lizenzerfüllende Notiz könnte sein:
\newline
\textit{Erstellt auf der Basis des CC-BY-4.0 lizenzierten Tools \texttt{proMusicologica} von K. Reincke \copyright{} 2022 [
Repository \href{https://github.com/kreincke/proMusicologica.ltx}{https://github.com/kreincke/proMusicologica.ltx} ,
Lizenztext \href{https://creativecommons.org/licenses/by/4.0/}{https://creativecommons.org/licenses/by/4.0/} ]
}}
}

\maketitle

\begin{abstract}
\noindent \itshape
Dies ist ein \LaTeX, Bib\LaTeX, \emph{proMusicologica.ltx} und \emph{Lilypond} basiertes Template für leicht zu scheibende musikwissenschaftliche Artikel. Es demonstriert seinen Zweck mit einer Zitatverifikation und integrierten Notenbeispielen -- nur, damit Sie das durch Ihren eigenen Text ersetzen.
\end{abstract}

\footnotesize
\tableofcontents
\normalsize

\section{Worum es geht.}
Die Geschichte der Musikwissenschaft als Wissenschaft muss noch geschrieben werden.

Natürlich: schon eine einfache Suche mit 'Google Scholar' listet Texte auf, die diesem Thema nachzugehen vorgeben. Doch detailliert betrachtet, verhandeln sie die Musikwissenschaft als Fach von ihrem Gegenstand oder von der einer Teildisziplin spezifischen Methodik her, nicht aber von ihrer spezifischen Wissenschaftlichkeit her. Das zeigen auch mehr oder minder aktuelle Einführung in die Musikwissenschaft.

Die angehende Musikwissenschaftlerin muss das bedrängen. Denn natürlich ist nicht jede Arbeit, die sich mit a b c neschäftigt, schon allein deswegen eine musikwissenschaftliche Arbeit. Es muss also zusätzliche Kriterien geben, die ihre Texten erfüllen müssen, wenn sie musikwissenschaftliche Arbeiten sein wollen. Schön wäre es, wenn die Musikwissenschaft selbst sich ihre Wissenschaftlichkeit versichert, diese sekundär Kriterien offenlegt und deren Wandel über die Zeit dokumentiert. Das wäre dann eine Geschichte der Musikwissenschaft als Wissenschaft.

Bis dahin beerbt die Musikwissenschaft in dieser Hinsicht Nachbardisziplinen.

Das Bedrängende für die  angehende Musikwissenschaftlerin besteht konkret darin, dass sie sich einen Set von Tools für Kriterien zusammenstellen muss, die ihr von ihrer Disziplin her noch nicht systematisch angeboten werden.

Hier möchte die vorliegende Arbeit einspringen: Sie will nicht das Desiderat beschreiben und als Desiderat belegen. Sie will vielmehr ein Tool für das musikwissenschatliche Schreiben vorstellen, dass den Schreibstil der Alt-Philologie, der Philosphie und der Geschichtswissenschaft auf die Musikwissenschaft begründet überträgt.

Natürlich wissen wir, dass es in der Musikwissenschaft andere Stile gibt, ... So soll dieser Text auch begründen, warum wir diesen einen vorziehen. Insgesamt darf sie also auch als ein selbszreferentielles Tutorial gelesen werden: sie zeigt, wie es sein sollte, und erlaubt über die Einsicht in den Quellcode einen Blick, wie es technisch umgesetzt wird.

\section{Was der Zweck ist.}

\section{Wie es umgesetzt wird.}


\section{Kontextsensitiver Zitattest}

\begin{itemize}
  \item \enquote{Alle Daten} = 1. Initialzitat\footcite[vgl.][123]{Grabner1974a}
  \item \enquote{ders. ebda} = dasselbe Werk, dieselbe Seite\footcite[vgl.][123]{Grabner1974a}
  \item \enquote{ders. a.a.O. p.} = dasselbe Werk, andere Seite\footcite[vgl.][125f]{Grabner1974a}
  \item \enquote{Alle Daten} = 2. Initialzitat\footcite[vgl.][123]{Delamotte2011a}
  \item \enquote{Kurztitel} = erstes Werk, erste Seite, anderer  Kontext\footcite[vgl.][123]{Grabner1974a}
\end{itemize}

\section{Musikintegrationstest}

Inline-Symbole in der Zeile = \Takt{4}{4} \SePa \Vier\SechBL\SechBL\ \Halb\Pu\ $|$ \HH.T..... \HH.S.3.... \HH.D.3.9.7.. \HH.T....., danach ein Lilypond basiertes Notenbeispiel:

\begin{center}
\begin{lilypond}
\version "2.18.2"

\header { tagline = "" }
\include "lilypond/harmonyli.ly"

\score {
  \new StaffGroup {
    \time 4/4
    <<
      \new Staff {
        \relative d' {
       	  \clef "treble" \key d \major  \stemUp
          r1 | < fis a d>1 \bar "||"
        }
        \addlyrics {
          \markup \setHas "T" #'(("" . " "))
        }
      }
      \new Staff {
        \relative d {
          \clef "bass" \key d \major \stemDown
          r4 d4 fis4 a | <d, d'>1 \bar "||"
        }
      }
    >>
  }
  \layout {
    \context { \Lyrics \consists "Text_spanner_engraver" }
  }
  \midi {}
}

\end{lilypond}
\end{center}


% This file is part of proMusicologica.ltx
% (c) 2022 Karsten Reincke (https://github.com/kreincke/proMusicologica.ltx)
% It is distributed under the terms of the creative commons license
% CC-BY-4.0 (= https://creativecommons.org/licenses/by/4.0/)


% specific abbreviations
\abbr[utb]{UTB}{Uni-Taschenbuch}
\abbr[stw]{stw}{suhrkamp taschenbuch wissenschaft}

% general abbreviations
\abbr[vgl]{vgl.}{vergleiche}
\abbr[aaO]{a.a.O.}{am angegebenen Ort}
\abbr[ebda]{ebda.}{ebenda}
\abbr[ifross]{ifross}{Institut für Rechtsfragen der Freien und Open Source Software}
\abbr[wp]{wp.}{webpage = Webdokument ohne innere Seitennummerierung}
\abbr[RDL]{RDL}{Referenzdownload am}

% This file is part of proMusicologica.ltx
% (c) 2022 Karsten Reincke (https://github.com/kreincke/proMusicologica.ltx)
% It is distributed under the terms of the creative commons license
% CC-BY-4.0 (= https://creativecommons.org/licenses/by/4.0/)

\abbr[fzmw]{FZMw}{Frankfurter Zeitung für Musikwissenschaft}

\abbr[afmw]{AfMW}{Archiv für Musikwissenschaft [ISSN: 0003-9292]}
\abbr[dtk]{}{Die Tonkunst [ISSN: 1863-3536]}

\printnomenclature
\printbibliography

\end{document}
