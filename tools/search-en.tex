% This file is part of proMusicologica.ltx
% (c) 2022 Karsten Reincke (https://github.com/kreincke/proMusicologica.ltx)
% It is distributed under the terms of the creative commons license
% CC-BY-4.0 (= https://creativecommons.org/licenses/by/4.0/)

\documentclass[
  DIV=calc,
  BCOR=5mm,
  11pt,
  headings=small,
  oneside,
  abstract=true,
  toc=bib,
  ngerman,english]{scrartcl}

%%% (1) general configurations %%%
\usepackage[utf8]{inputenc}
\usepackage{a4}
\usepackage[ngerman,english]{babel}

\usepackage[
  backend=biber,
  style=authortitle-dw,
  sortlocale=auto,
]{biblatex}
\input{../source/cfg/inc.cfg-biber-en.tex}

\addbibresource{../source/bib/lit.verify.bib}
%\addbibresource{../source/bib/lit.main.bib}

% package for improving the grey value and the line feed handling
\usepackage{microtype}

%language specific quoting signs
\usepackage[
  style=german,
  autostyle=true,
]{csquotes}

% language specific hyphenation
% This file is part of proMusicologica.ltx
% (c) 2022 Karsten Reincke (https://github.com/kreincke/proMusicologica.ltx)
% It is distributed under the terms of the creative commons license
% CC-BY-4.0 (= https://creativecommons.org/licenses/by/4.0/)

\hyphenation{ pro-scien-tia there-fo-re}

\babelhyphenation{ pro-scien-tia wor-tmi-tfalsch-entr-ennu-ngen}


%%% (3) layout page configuration %%%

% select the visible parts of a page
% S.31: { plain|empty|headings|myheadings }
\pagestyle{headings}

% select the wished style of page-numbering
% S.32: { arabic,roman,Roman,alph,Alph }
\pagenumbering{arabic}
\setcounter{page}{1}

% select the wished distances using the general setlength order:
% S.34 { baselineskip| parskip | parindent }
% - general no indent for paragraphs
\setlength{\parindent}{0pt}
\setlength{\parskip}{1.2ex plus 0.2ex minus 0.2ex}


%- start(footnote-configuration)

\deffootnote[1.5em]{1.5em}{1.5em}{\textsuperscript{\thefootnotemark)\ }}

%integrate nomenclature
\input{../source/cfg/inc.cfg-ncl-en.tex}

% Hyperlinks
\usepackage{hyperref}
\hypersetup{bookmarks=true,breaklinks=true,colorlinks=true,citecolor=blue,draft=false}

\usepackage[dvips]{epsfig}

% graphic
\usepackage{graphicx,color}
\usepackage{array}
\usepackage{shadow}
\usepackage{fancybox}


%- end(footnote-configuration)

% package for macking tables with broken lines
\usepackage{multirow}

%for using label as nameref
\usepackage{nameref}


\newcommand{\LibraryCatalogA}{Library Catalog A }
\newcommand{\LibraryCatalogB}{Library Catalog B }
\newcommand{\LibraryCatalogC}{Library Catalog C }
\newcommand{\LibraryCatalogD}{Library Catalog D }
\newcommand{\LibraryCatalogE}{Library Catalog E }
\newcommand{\LibraryCatalogF}{Library Catalog F }
\newcommand{\LibraryCatalogG}{Library Catalog G }
\newcommand{\LibraryCatalogH}{Library Catalog H }
\newcommand{\LibraryCatalogI}{Library Catalog I }
\newcommand{\LibraryCatalogJ}{Library Catalog J }
\newcommand{\LibraryCatalogK}{Library Catalog K }
%\newcommand{\LibraryCatalogL}{Library Catalog A}
%\newcommand{\LibraryCatalogM}{Library Catalog A}
%\newcommand{\LibraryCatalogN}{Library Catalog A}
%\newcommand{\LibraryCatalogO}{Library Catalog A}
%\newcommand{\LibraryCatalogP}{Library Catalog A}
\newcommand{\LibraryCatalogX}{Library Catalog ?}

\newcommand{\SearchItemA}{Search Item A}
\newcommand{\SearchItemB}{Search Item B}
\newcommand{\SearchItemC}{Search Item C}
\newcommand{\SearchItemD}{Search Item D}
\newcommand{\SearchItemE}{Search Item E}
\newcommand{\SearchItemF}{Search Item F}
\newcommand{\SearchItemG}{Search Item G}
\newcommand{\SearchItemH}{Search Item H}
\newcommand{\SearchItemI}{Search Item I}
\newcommand{\SearchItemJ}{Search Item J}
\newcommand{\SearchItemK}{Search Item L}
\newcommand{\SearchItemL}{Search Item M}

% Marker for a catalog classification
\newcommand{\NecessaryCatalog}{$\clubsuit$}
\newcommand{\NecessaryCatalogDefinition}{necessary catalog}
\newcommand{\ImportantCatalog}{$\spadesuit$}
\newcommand{\ImportantCatalogDefinition}{important catalog}
\newcommand{\UsefulCatalog}{$\heartsuit$}
\newcommand{\UsefulCatalogDefinition}{useful catalog}
\newcommand{\PossibleCatalog}{$\diamondsuit$}
\newcommand{\PossibleCatalogDefinition}{potential catalog}

% Marker for a status reportin
\newcommand{\many}{$\ast$}
\newcommand{\manyDef}{found many hits}
\newcommand{\some}{$\star$}
\newcommand{\someDef}{found some hits}
\newcommand{\few}{$\odot$}
\newcommand{\fewDef}{found a few hits}
\newcommand{\nothing}{$\neg$}
\newcommand{\nothingDef}{didn't found any hits}
\newcommand{\ongoing}{$\circ$}
\newcommand{\ongoingDef}{current evaluation}
\newcommand{\open}{?}
\newcommand{\openDef}{future evaluation}
\newcommand{\ignored}{-}
\newcommand{\ignoredDef}{canceled evaluation}


%% use all entries of the bliography

\begin{document}


\begin{table}
\scriptsize
\caption{Resources of the University Library XYZ}
\begin{center}
\begin{tabular}[h]{|r|c|c|c||c||c|c|c|c||c|c|c|c|c|c|c|c||c|}
\hline
& \rotatebox{90}{$\clubsuit$ \textit{\LibraryCatalogA}}
& \rotatebox{90}{$\clubsuit$ \textit{\LibraryCatalogB}}
& \rotatebox{90}{$\clubsuit$ \textit{\LibraryCatalogC}}
& \rotatebox{90}{$\spadesuit$ \textit{\LibraryCatalogD}}
& \rotatebox{90}{$\heartsuit$ \textit{\LibraryCatalogE}}
& \rotatebox{90}{$\heartsuit$ \textit{\LibraryCatalogF}}
& \rotatebox{90}{$\heartsuit$ \textit{\LibraryCatalogG}}
& \rotatebox{90}{$\heartsuit$ \textit{\LibraryCatalogH}}
& \rotatebox{90}{$\diamondsuit$ \textit{\LibraryCatalogI}}
& \rotatebox{90}{$\diamondsuit$ \textit{\LibraryCatalogJ}}
& \rotatebox{90}{$\diamondsuit$ \textit{\LibraryCatalogK}}
& \rotatebox{90}{\textit{\LibraryCatalogX}}
\\
\hline \hline
\SearchItemA
  & \many & \some & \few & \nothing & \ongoing & \open
  & \ignored & ? & ? & ? & ? & ?\\
\hline
\SearchItemB
  & ? & ? & ? & ? & ? & ?
  & ? & ? & ? & ? & ? & ?\\
\hline
\SearchItemC
  & ? & ? & ? & ? & ? & ?
  & ? & ? & ? & ? & ? & ?\\
\hline
\SearchItemD
  & ? & ? & ? & ? & ? & ?
  & ? & ? & ? & ? & ? & ?\\
\hline
\SearchItemE
  & ? & ? & ? & ? & ? & ?
  & ? & ? & ? & ? & ? & ?\\
\hline
\SearchItemF
  & ? & ? & ? & ? & ? & ?
  & ? & ? & ? & ? & ? & ?\\
\hline
\SearchItemG
  & ? & ? & ? & ? & ? & ?
  & ? & ? & ? & ? & ? & ?\\
\hline
\SearchItemH
  & ? & ? & ? & ? & ? & ?
  & ? & ? & ? & ? & ? & ?\\
\hline
\SearchItemI
  & ? & ? & ? & ? & ? & ?
  & ? & ? & ? & ? & ? & ?\\
\hline
\SearchItemJ
  & ? & ? & ? & ? & ? & ?
  & ? & ? & ? & ? & ? & ?\\
\hline
\SearchItemK
  & ? & ? & ? & ? & ? & ?
  & ? & ? & ? & ? & ? & ?\\
\hline
\SearchItemL
  & ? & ? & ? & ? & ? & ?
  & ? & ? & ? & ? & ? & ?\\
\hline
\hline

\end{tabular}
\end{center}
\end{table}

\textbf{Literature research} in the library of the university
\emph{XYZ}\footcite[s.][n.P.]{UbXYZ2018a}:


\begin{description}
  \item[Status of the Catalogues] :-
    \begin{description}
      \item[\NecessaryCatalog] :- \NecessaryCatalogDefinition
      \item[\ImportantCatalog] :- \ImportantCatalogDefinition
      \item[\UsefulCatalog] :- \UsefulCatalogDefinition
      \item[\PossibleCatalog] :- \PossibleCatalogDefinition
    \end{description}
  \item[Status of the report] :-
    \begin{description}
      \item[\many] :- \manyDef
      \item[\some] :- \someDef
      \item[\few] :- \fewDef
      \item[\nothing] :- \nothingDef
      \item[\ongoing] :- \ongoingDef
      \item[\open] :- \openDef
      \item[\ignored] :- \ignoredDef
    \end{description}
\end{description}



\end{document}
