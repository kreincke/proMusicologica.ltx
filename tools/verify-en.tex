% This file is part of proMusicologica.ltx
% (c) 2022 Karsten Reincke (https://github.com/kreincke/proMusicologica.ltx)
% It is distributed under the terms of the creative commons license
% CC-BY-4.0 (= https://creativecommons.org/licenses/by/4.0/)

\documentclass[
  DIV=calc,
  BCOR=5mm,
  11pt,
  headings=small,
  oneside,
  abstract=true,
  toc=bib,
  ngerman,english]{scrartcl}

%%% (1) general configurations %%%
\usepackage[utf8]{inputenc}
\usepackage{a4}
\usepackage[ngerman,english]{babel}

\usepackage[
  backend=biber,
  style=authortitle-dw,
  sortlocale=auto,
]{biblatex}
\input{../source/cfg/inc.cfg-biber-en.tex}

\addbibresource{../source/bib/lit.verify.bib}
\addbibresource{../source/bib/lit.main.bib}

% package for improving the grey value and the line feed handling
\usepackage{microtype}

%language specific quoting signs
\usepackage[
  style=german,
  autostyle=true,
]{csquotes}

% language specific hyphenation
% This file is part of proMusicologica.ltx
% (c) 2022 Karsten Reincke (https://github.com/kreincke/proMusicologica.ltx)
% It is distributed under the terms of the creative commons license
% CC-BY-4.0 (= https://creativecommons.org/licenses/by/4.0/)

\hyphenation{ pro-scien-tia there-fo-re}

\babelhyphenation{ pro-scien-tia wor-tmi-tfalsch-entr-ennu-ngen}


%%% (3) layout page configuration %%%

% select the visible parts of a page
% S.31: { plain|empty|headings|myheadings }
\pagestyle{headings}

% select the wished style of page-numbering
% S.32: { arabic,roman,Roman,alph,Alph }
\pagenumbering{arabic}
\setcounter{page}{1}

% select the wished distances using the general setlength order:
% S.34 { baselineskip| parskip | parindent }
% - general no indent for paragraphs
\setlength{\parindent}{0pt}
\setlength{\parskip}{1.2ex plus 0.2ex minus 0.2ex}


%- start(footnote-configuration)

\deffootnote[1.5em]{1.5em}{1.5em}{\textsuperscript{\thefootnotemark)\ }}

%for using label as nameref
\usepackage{nameref}

%integrate nomenclature
\input{../source/cfg/inc.cfg-ncl-en.tex}

% Hyperlinks
\usepackage{hyperref}
\hypersetup{bookmarks=true,breaklinks=true,colorlinks=true,citecolor=blue,draft=false}

\begin{document}

%% use all entries of the bliography
\nocite{*}

%%-- start(titlepage)
\titlehead{Bib\LaTeX}
\subject{Release 1.0}
\title{Testing Bibliographic Data\footnote{
Developed on the base of the \texttt{CC-BY-4.0} licensed tool \textit{proMusicologica.ltx} by Karsten Reincke \copyright{} 2022 [
repository = \href{https://github.com/kreincke/proMusicologica.ltx}{https://github.com/kreincke/proMusicologica.ltx},
license text = \href{https://creativecommons.org/licenses/by/4.0/}{https://creativecommons.org/licenses/by/4.0/} ]}
}

\maketitle
%%-- end(titlepage)
\begin{abstract}
\noindent \itshape
This texts allows to verify the quality of bibliographic reference data. An entry of \texttt{lit.main.bib} is tested in different contexts:
\end{abstract}


%%%%%%%%%%%%%%%%%%%%%%%%%%%%%%%%%%%%%%%%%%%%%%%%%%%%%%%%%%%%%%%%%%%%%%%%
% Replace KantKdrV1974 by the BibtexKey of the dataset you want to test %
%%%%%%%%%%%%%%%%%%%%%%%%%%%%%%%%%%%%%%%%%%%%%%%%%%%%%%%%%%%%%%%%%%%%%%%%

\section{Context-Sensitive Test}
\begin{itemize}
  \item \enquote{complete data} = First initial quotation\footcite[vgl.][123]{Grabner1974a}
  \item \enquote{idem ibid.} = same work, same page as before\footcite[cf.][123]{Grabner1974a}
  \item \enquote{idem loc. cit., p.} = same work as before, different page\footcite[cf.][125f]{Grabner1974a}
  \item \enquote{complete data} = Second initial quotation\footcite[vgl.][123]{Delamotte2011a}
  \item \enquote{Shorttitel} = first work, first page, different context\footcite[cf.][123]{Grabner1974a}
\end{itemize}

% This file is part of proMusicologica.ltx
% (c) 2022 Karsten Reincke (https://github.com/kreincke/proMusicologica.ltx)
% It is distributed under the terms of the creative commons license
% CC-BY-4.0 (= https://creativecommons.org/licenses/by/4.0/)


\abbr[cf]{cf.}{confer}
\abbr[id]{id.}{idem = latin for 'the same', be it a man, woman or a group\ldots}
\abbr[ibid]{ibid.}{ibidem = latin for 'at the same place'}
\abbr[lc]{l.c.}{loco citato = latin for 'in the place cited'}
\abbr[wpe]{wp.}{webpage = Web document without a finer page division}
\abbr[RDL]{RDL}{Reference download on}

% This file is part of proMusicologica.ltx
% (c) 2022 Karsten Reincke (https://github.com/kreincke/proMusicologica.ltx)
% It is distributed under the terms of the creative commons license
% CC-BY-4.0 (= https://creativecommons.org/licenses/by/4.0/)

\abbr[fzmw]{FZMw}{Frankfurter Zeitung für Musikwissenschaft}

\abbr[afmw]{AfMW}{Archiv für Musikwissenschaft [ISSN: 0003-9292]}
\abbr[dtk]{}{Die Tonkunst [ISSN: 1863-3536]}

\printnomenclature
\printbibliography

\end{document}
